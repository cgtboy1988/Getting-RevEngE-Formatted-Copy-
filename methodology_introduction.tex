Our goal is to collect and analyze data about the strategies employed by reverse engineers. This data can be used to construct behavioral models which, in turn, can be used to evaluate the strength of particular obfuscating transformations against a variety of attacks. Ultimately, these insights can inform the design of new and better transformations and guide the deployment of current transformations.

To collect and analyze behavioral data, we must address 5 problems, detailed below.
%First, we need to be able to generate a large number of challenge problems of different complexity. Second, we need to recruit a sufficiently large sample of diverse subjects and provide incentives that will motivate them to invest time to solve the challenges. Third, we need to provide efficient and unobtrusive mechanisms to monitor the reverse engineers as they solve a challenge, and collect their behavioral data. Fourth, we need to be able to analyze the correctness of a de-obfuscated solution provided by the reverse engineer, i.e. we need to determine whether they have solved the challenge or not. Fifth, and finally, we need ways to analyze, summarize, and present the potentially vast behavioral data that has been collected. 

%We discuss these issues in some detail next.
