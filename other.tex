\section{Other Uses for Data Collection} \label{otheruses}

Although the RevEngE data collection suite was built for this particular application, its focus on the human-device interaction has pertinence to many other fields and problems.  In particular, this suite may be used in evaluating software, monitoring devices and users for security contexts, and examining how humans solve other problems, among other potential uses.

\subsection{Software Evaluation}

The goal of software testing consists of evaluating how well users leverage the given software to accomplish particular tasks.  In prior work, test subjects typically are given a device with some software, perhaps trained how to use that software, and are given tasks to accomplish with the software.  Typically, the variable of time is used to determine efficacy, and often evaluations compare one piece of software to others which seek to complete similar or identical tasks.~\footnote{For an overview of this type of methodology applied to data visualization software, see ~\cite{plaisant2004challenge}.}

However, this methodology can only determine if given software \textit{can be used to solve a problem} and how well the software performs \textit{compared to other enumerated pieces of software}.  The performance of the software compared to the population of other methods to complete the given tasks is not known; other software (or, perhaps even non-software methods) unknown to the researchers may perform better than any included in the study.

The data collection here has the potential to examine these unknown methods.  In such an experiment, the researchers might introduce subjects to the evaluation piece of software, give them tasks, and have them complete these tasks while monitoring them with the data collection software.  Because the data collection contains information regarding all software a user runs, researchers may evaluate all of the methods subjects employed on their devices to complete the tasks, including the software introduced by the researchers and perhaps methods known to the subjects but not the researchers.  Additionally, this method may be able to reduce costs by enabling distribution of the experiment; where other methods require a closed environment in order to ensure subjects utilize particular software to solve problems, the data collection suite negates such a requirement by recording the exact software subjects run when completing a task.

The data collection software may also augment other evaluation methodologies.  When run in typical software evaluation experiments, the data collected via the RevEngE suite offers information regarding what software users run concurrently with the experiment software to boost (or detract from) efficacy and records of user input which can aid in analyzing what, within the experimental software, users do, in turn revealing, for instance, software features which might be ineffective.

\subsection{Security Monitoring}

Many entities, such as companies, government agencies, and educational institutions, operate large scale networks with many users and devices.  Monitoring these devices for security purposes allows prevention, detection, and response to security incidents~\cite{kent2006guide}.  Operators of these networks currently actively monitor network traffic in order to accomplish these goals~\cite{binde2011assessing}.  Concurrently, antivirus software may monitor various local resources on devices.

The data collection suite built for RevEngE may expand on these efforts by offering a new type of data to analyze:  User-device interaction collected from users/device pairs on the network.  This, in turn, enables analysis of what otherwise authorized users do on these devices.  Such data could, as a potential example, reveal man-at-the-end attacks by monitor users running typical reverse engineering tools.

\subsection{Human-Device Problem Solving}

As a general matter, humans employ devices to solve many different problems.  From writing documents to viewing maps to engaging in social media to making music to reverse engineering and programming, humans employ software to accomplish different tasks.  Moreover, humans use many different combinations of software in intricate ways to accomplish these types of tasks.  However, it is not well known what software (and combinations thereof) are most prevalent or most for particular tasks---the research here analyzing reverse engineering is a subset of this phenomenon.  The data collection suite included in RevEngE may be employed to study this interaction in many different contexts using methods that parallel the examination of obfuscation and reverse engineering here.

Currently, efforts are underway to employ the data collection suite in computer security capture-the-flag competitions~\cite{taylor2017ctf}.  This deployment aims to examine security problems similar in nature to the reverse engineering case:  By analyzing data from participants solving complicated and hard computer security problems, the solvability and methods used to achieve that solvability may be approximated, depending on the quality of the subjects.  In addition, the data collected may be employed to aid in evaluating subjects' abilities, as well the effect instruction has on those abilities.
